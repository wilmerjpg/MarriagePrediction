%%%%%%%%%%%%%%%%%%%%%%%%%%%%%%%%%%%%%%%%%%%%%%%%%%%%%%%%%%%%%%%%%%%%%%
% writeLaTeX Example: Academic Paper Template
%
% Source: http://www.writelatex.com
% 
% Feel free to distribute this example, but please keep the referral
% to writelatex.com
% 
%%%%%%%%%%%%%%%%%%%%%%%%%%%%%%%%%%%%%%%%%%%%%%%%%%%%%%%%%%%%%%%%%%%%%%
% How to use writeLaTeX: 
%
% You edit the source code here on the left, and the preview on the
% right shows you the result within a few seconds.
%
% Bookmark this page and share the URL with your co-authors. They can
% edit at the same time!
%
% You can upload figures, bibliographies, custom classes and
% styles using the files menu.
%
% If you're new to LaTeX, the wikibook is a great place to start:
% http://en.wikibooks.org/wiki/LaTeX
%
%%%%%%%%%%%%%%%%%%%%%%%%%%%%%%%%%%%%%%%%%%%%%%%%%%%%%%%%%%%%%%%%%%%%%%
\documentclass[twocolumn,showpacs,%
  nofootinbib,aps,superscriptaddress,%
  eqsecnum,prd,notitlepage,showkeys,10pt]{revtex4-1}

\usepackage{amssymb}
\usepackage[spanish]{babel}							%Incluye caracteres del español
\usepackage{dcolumn}
\usepackage{hyperref}
\usepackage[utf8]{inputenc}							%Permite añadir tildes al documento
\usepackage{lipsum}

\begin{document}

\title{Proyecto Predicción de Estado Civil}
\author{Nobrega Yanelly y Prieto Wilmer}
\email{yanellynobrega@gmail.com, wilmerprieto3000@gmail.com}
\affiliation{Escuela de Computación, Facultad de Ciencias, \\ Universidad Central de Venezuela \\ Caracas, Venezuela}


\begin{abstract}
Resumen -
Este proyecto está basado en la búsqueda de un algoritmo que permita generar un modelo que prediga de manera eficiente el estado civil de una pareja dado un set de datos el cual contiene una serie de caracteísticas de la pareja, tales como: la edad del hombre, edad de la mujer, ingreso anual de ambos, entre otros. Todos estos datos pasarán por un preprocesamiento y finalmente para la generación de los modelos se utilizará Arbol de Decisión, K-Vecinos y Reglas de Clasificación.   
\\ \\
Palabras Clave: predicción, pre-procesamiento, clasificación, estado civil, pareja, arbol de decision, K-Vecinos.
\end{abstract}

\maketitle

\section{Introducción}
Exixten diversos algoritmos de clasificación  basados en un conjunto de calculos y reglas que permiten crear un modelo a partir de los datos. Para la creación de estos modelos, el algoritmo analiza los datos proporcionados en busca de tipos específicos de patrones o tendencias y usa los resultados de este análisis para definir los parámetros óptimos para la creación del modelo. A continuación, estos parámetros se aplican en  un conjunto de datos de entrenamiento para extraer patrones procesables. 

Para este proyecto utilizaremos 3 algoritmos de clasificación (Arbol de Decisión,  K-Vecinos y Reglas de Clasificación) con el fin de predecir eficientemente el estado civil de una pareja dado un set de datos con caracteristicas que determinan si la pareja esta separada, casada o divorciada.

\section{Explicación del Problema}
Se plantea el objetivo de predecir el estado civil de una pareja dado un set de datos. Dicho set de datos posee siete columnas que son: \textbf{ID} - Identificador unico para cada registro(pareja), \textbf{GAGE} - Edad de la Mujer, \textbf{BAJE} - Edad del Hombre, \textbf{GP}, \textbf{BP}, \textbf{AINCOME} - Salario Anual en dolares percibido por la pareja y \textbf{STATUS} - Estado civil de la pareja(Atributo a predecir). Ya identificadas las columnas se procederá con el siguiente paso que es el preprocesamiento de los datos.

\section{Preprocesamiento de Datos}
Para realizar el preprocesamiento de la data se realizó un análisis exploratorio de los mismos en donde se observó que existian columnas irrelevantes en cuanto al objetivo del problema, dichas columnas fueron el ID y AINCOME los cuales representan el identificador de la fila y el sueldo anual de la pareja respectivamente. En cuanto a este último atributo lo consideramos irrelevante ya que no hallamos ningún tipo de patrón que nos encaminara a predecir el valor real del estatus de la pareja. Seguido de esto se decidió aplicar una numerización a las columnas que contenían variables nominales las cuales podian asumir n valores, dichas columnas son: GP, BP y STATUS. Cabe destacar que al realizar el análisis se pudo concluir que la diferencia entre ambas edades era un factor fuerte para determinar el estado civil debido a que se observaron diversas similitudes de dieferencia de edad para un mismo valor de clase, por lo que se procedio a eliminar ambas columnas y crear una nueva almacenando la diferencia de edades.  

\section{Modelos de Clasificación}
Los algoritmos utilizados para lograr predecir el estado civil de una pareja con los datos obtenidos y ya preprocesados fueron Arbol de Decision, K-Vecinos y Reglas de Clasificacion. Se dicidió utilizar estos métodos de clasificación porque fueron los estudiados en la materia de Minería de Datos y de los cuales se tenía mayor conocimiento a la hora de implementar la solución al problema planteado. Además es bueno destacar que son métodos con un funcionamiento facil de entender, y dado que cada algoritmo de estos se basa en diferentes principios, permite que dependiendo del problema algunos modelos sean mejores que otros, ampliando así la posibilidad de lograr un buen resultado.  

\subsection{Árbol de Decisión}
Se generó el modelo para el árbol de decisión usando la función rpart,  cuyos parametros fueron: (STATUS ~.) el cual denota que se quiere predecir la característica STATUS  (estado civil) de la pareja basándose en las demás características provistas en el set de datos y el parámetro method: (class) indicando se quiere un modelo de clasificación. En cuanto a los parámetros de control: minsplit, cp, maxdepth (los cuales nos indican la cantidad mínima de individuos por nodo, el grado de brusquedad en los cambios y la profundidad de árbol respectivamente)  no fueron utilizados debido a que en al realizar la permutación de los mismos, los resultados del arbol fueron muy similares. Para la evaluación de este modelo se tomó en cuenta la matriz de confusión y el area bajo la curva (función roc del paquete pROC) obteniendo como resultado un noventa (90) por ciento de aciertos y 0.97 de area bajo la curva respectivamente. 

\subsection{K-Vecinos}
Se generó el modelo de K-Vecinos haciendo uso de la función knn, pasando los conjuntos de prueba y entrenamiento, y el parámetro  k (número de vecinos considerados) aginándole el valor 28 debido a que se tomó como medida la raíz cuadrada del número total de individuos en el set de datos (750) que suele ser la medida que mejor se comporta teóricamente. Para la evaluación de este modelo se tomó en cuenta la matriz de confusión y el área bajo la curva (función roc del paquete pROC) obteniendo como resultado un cincuenta (50) por ciento de aciertos y 0.62 de area bajo la curva respectivamente. 

\subsection{Reglas de Clasificación}
Se generó el modelo de Reglas de Clasificación haciendo uso de la función JRip provista por la interfaz RWeka, para ello debimos realizar una transformación de los datos haciendo uso de \textbf{as.factor}, y luego aplicamos dicha función JRip, donde el primer parámetro es formula: (STATUS ~ .)  el cual denota que se quiere predecir la característica STATUS de la pareja basándose en las demás características provistas en el set de datos, y el segundo parámetro que indica la data a utilizar que en este caso es la data de entrenamiento.  Para la evaluación de este modelo se tomó en cuenta la matriz de confusión y el area bajo la curva (función roc del paquete pROC) obteniendo como resultado un cien (100) por ciento de aciertos y 1 de area bajo la curva respectivamente. 

\section{Conclusión}
Como mencionamos anteriormente para realizar la comparación de los resultados obtenidos entre los tres modelos implementados, se tomó en cuenta la matriz de confusión para obtener la tasa de aciertos y el area bajo la curva (función roc del paquete pROC) de cada uno de ellos. Se tiene que el mejor modelo obtenido es el basado en Reglas de Clasificación debido a que la tasa de aciertos es perfecta al igual que el area bajo la curva, es decir, para ambos casos el resultado fue de 1. Seguidamente el modelo basado en árbol de desición con una tasa de aciertos de 0.91 y con un área bajo la curva de 0.97. Y por último se tiene el módelo de K-Vecinos el cual su tasa de aciertos fue de 0.5 y el area bajo la curva de 0.62 por lo que queda totalmente descartado como mejor modelo. Sin duda alguna el mejor modelo para el caso estudiado es el modelo basado en Reglas de Clasficación.

\section{Agradecimientos}
Agradecemos al Profesor Fernando Crema, quien se encargo de instruirnos en el área de Minería de Datos especificamente en lo refrente a reglas de clasificacion lo cual nos ayudó a elaborar dicho estudio y también al preparador de Minería de Datos Wilmer Gonzalez quien estuvo atento a cualquier duda que se nos presto para ayudarnos.
    
\section{Referencias Bibliográficas}

\begin{enumerate}

	\item Marriage Prediction, Kaggle Competitions. Disponible en: \url{https://inclass.kaggle.com/c/marriage-prediction/}
    
\end{enumerate}


\end{document}